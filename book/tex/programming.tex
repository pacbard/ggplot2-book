\hypertarget{cha:programming}{%
\chapter{Programming with ggplot2}\label{cha:programming}}

\hypertarget{introduction}{%
\section{Introduction}\label{introduction}}

A major requirement of a good data analysis is flexibility. If your data
changes, or you discover something that makes you rethink your basic
assumptions, you need to be able to easily change many plots at once.
The main inhibitor of flexibility is code duplication. If you have the
same plotting statement repeated over and over again, you'll have to
make the same change in many different places. Often just the thought of
making all those changes is exhausting! This chapter will help you
overcome that problem by showing you how to program with ggplot2.
\index{Programming}

To make your code more flexible, you need to reduce duplicated code by
writing functions. When you notice you're doing the same thing over and
over again, think about how you might generalise it and turn it into a
function. If you're not that familiar with how functions work in R, you
might want to brush up your knowledge at
\url{http://adv-r.had.co.nz/Functions.html}.

In this chapter I'll show how to write functions that create:

\begin{itemize}
\tightlist
\item
  A single ggplot2 component.
\item
  Multiple ggplot2 components.
\item
  A complete plot.
\end{itemize}

And then I'll finish off with a brief illustration of how you can apply
functional programming techniques to ggplot2 objects.

You might also find the
\href{https://github.com/wilkelab/cowplot}{cowplot} and
\href{https://github.com/jrnold/ggthemes}{ggthemes} packages helpful. As
well as providing reusable components that help you directly, you can
also read the source code of the packages to figure out how they work.

\hypertarget{single-components}{%
\section{Single components}\label{single-components}}

Each component of a ggplot plot is an object. Most of the time you
create the component and immediately add it to a plot, but you don't
have to. Instead, you can save any component to a variable (giving it a
name), and then add it to multiple plots:

\begin{Shaded}
\begin{Highlighting}[]
\NormalTok{bestfit <-}\StringTok{ }\KeywordTok{geom_smooth}\NormalTok{(}
  \DataTypeTok{method =} \StringTok{"lm"}\NormalTok{, }
  \DataTypeTok{se =} \OtherTok{FALSE}\NormalTok{, }
  \DataTypeTok{colour =} \KeywordTok{alpha}\NormalTok{(}\StringTok{"steelblue"}\NormalTok{, }\FloatTok{0.5}\NormalTok{), }
  \DataTypeTok{size =} \DecValTok{2}
\NormalTok{)}
\KeywordTok{ggplot}\NormalTok{(mpg, }\KeywordTok{aes}\NormalTok{(cty, hwy)) }\OperatorTok{+}\StringTok{ }
\StringTok{  }\KeywordTok{geom_point}\NormalTok{() }\OperatorTok{+}\StringTok{ }
\StringTok{  }\NormalTok{bestfit}
\KeywordTok{ggplot}\NormalTok{(mpg, }\KeywordTok{aes}\NormalTok{(displ, hwy)) }\OperatorTok{+}\StringTok{ }
\StringTok{  }\KeywordTok{geom_point}\NormalTok{() }\OperatorTok{+}\StringTok{ }
\StringTok{  }\NormalTok{bestfit}
\end{Highlighting}
\end{Shaded}

\begin{figure}[H]
  \centering
  \includegraphics[width=0.375\linewidth]{_figures/programming/layer9-1}%
  \includegraphics[width=0.375\linewidth]{_figures/programming/layer9-2}
\end{figure}

That's a great way to reduce simple types of duplication (it's much
better than copying-and-pasting!), but requires that the component be
exactly the same each time. If you need more flexibility, you can wrap
these reusable snippets in a function. For example, we could extend our
\texttt{bestfit} object to a more general function for adding lines of
best fit to a plot. The following code creates a \texttt{geom\_lm()}
with three parameters: the model \texttt{formula}, the line
\texttt{colour} and the line \texttt{size}:

\begin{Shaded}
\begin{Highlighting}[]
\NormalTok{geom_lm <-}\StringTok{ }\ControlFlowTok{function}\NormalTok{(}\DataTypeTok{formula =}\NormalTok{ y }\OperatorTok{~}\StringTok{ }\NormalTok{x, }\DataTypeTok{colour =} \KeywordTok{alpha}\NormalTok{(}\StringTok{"steelblue"}\NormalTok{, }\FloatTok{0.5}\NormalTok{), }
                    \DataTypeTok{size =} \DecValTok{2}\NormalTok{, ...)  \{}
  \KeywordTok{geom_smooth}\NormalTok{(}\DataTypeTok{formula =}\NormalTok{ formula, }\DataTypeTok{se =} \OtherTok{FALSE}\NormalTok{, }\DataTypeTok{method =} \StringTok{"lm"}\NormalTok{, }\DataTypeTok{colour =}\NormalTok{ colour,}
    \DataTypeTok{size =}\NormalTok{ size, ...)}
\NormalTok{\}}
\KeywordTok{ggplot}\NormalTok{(mpg, }\KeywordTok{aes}\NormalTok{(displ, }\DecValTok{1} \OperatorTok{/}\StringTok{ }\NormalTok{hwy)) }\OperatorTok{+}\StringTok{ }
\StringTok{  }\KeywordTok{geom_point}\NormalTok{() }\OperatorTok{+}\StringTok{ }
\StringTok{  }\KeywordTok{geom_lm}\NormalTok{()}
\KeywordTok{ggplot}\NormalTok{(mpg, }\KeywordTok{aes}\NormalTok{(displ, }\DecValTok{1} \OperatorTok{/}\StringTok{ }\NormalTok{hwy)) }\OperatorTok{+}\StringTok{ }
\StringTok{  }\KeywordTok{geom_point}\NormalTok{() }\OperatorTok{+}\StringTok{ }
\StringTok{  }\KeywordTok{geom_lm}\NormalTok{(y }\OperatorTok{~}\StringTok{ }\KeywordTok{poly}\NormalTok{(x, }\DecValTok{2}\NormalTok{), }\DataTypeTok{size =} \DecValTok{1}\NormalTok{, }\DataTypeTok{colour =} \StringTok{"red"}\NormalTok{)}
\end{Highlighting}
\end{Shaded}

\begin{figure}[H]
  \centering
  \includegraphics[width=0.375\linewidth]{_figures/programming/geom-lm-1}%
  \includegraphics[width=0.375\linewidth]{_figures/programming/geom-lm-2}
\end{figure}

Pay close attention to the use of ``\texttt{...}''. When included in the
function definition ``\texttt{...}'' allows a function to accept
arbitrary additional arguments. Inside the function, you can then use
``\texttt{...}'' to pass those arguments on to another function. Here we
pass ``\texttt{...}'' onto \texttt{geom\_smooth()} so the user can still
modify all the other arguments we haven't explicitly overridden. When
you write your own component functions, it's a good idea to always use
``\texttt{...}'' in this way. \indexc{...}

Finally, note that you can only \emph{add} components to a plot; you
can't modify or remove existing objects.

\hypertarget{exercises}{%
\subsection{Exercises}\label{exercises}}

\begin{enumerate}
\def\labelenumi{\arabic{enumi}.}
\item
  Create an object that represents a pink histogram with 100 bins.
\item
  Create an object that represents a fill scale with the Blues
  ColorBrewer palette.
\item
  Read the source code for \texttt{theme\_grey()}. What are its
  arguments? How does it work?
\item
  Create \texttt{scale\_colour\_wesanderson()}. It should have a
  parameter to pick the palette from the wesanderson package, and create
  either a continuous or discrete scale.
\end{enumerate}

\hypertarget{multiple-components}{%
\section{Multiple components}\label{multiple-components}}

It's not always possible to achieve your goals with a single component.
Fortunately, ggplot2 has a convenient way of adding multiple components
to a plot in one step with a list. The following function adds two
layers: one to show the mean, and one to show its standard error:

\begin{Shaded}
\begin{Highlighting}[]
\NormalTok{geom_mean <-}\StringTok{ }\ControlFlowTok{function}\NormalTok{() \{}
  \KeywordTok{list}\NormalTok{(}
    \KeywordTok{stat_summary}\NormalTok{(}\DataTypeTok{fun.y =} \StringTok{"mean"}\NormalTok{, }\DataTypeTok{geom =} \StringTok{"bar"}\NormalTok{, }\DataTypeTok{fill =} \StringTok{"grey70"}\NormalTok{),}
    \KeywordTok{stat_summary}\NormalTok{(}\DataTypeTok{fun.data =} \StringTok{"mean_cl_normal"}\NormalTok{, }\DataTypeTok{geom =} \StringTok{"errorbar"}\NormalTok{, }\DataTypeTok{width =} \FloatTok{0.4}\NormalTok{)}
\NormalTok{  )}
\NormalTok{\}}
\KeywordTok{ggplot}\NormalTok{(mpg, }\KeywordTok{aes}\NormalTok{(class, cty)) }\OperatorTok{+}\StringTok{ }\KeywordTok{geom_mean}\NormalTok{()}
\KeywordTok{ggplot}\NormalTok{(mpg, }\KeywordTok{aes}\NormalTok{(drv, cty)) }\OperatorTok{+}\StringTok{ }\KeywordTok{geom_mean}\NormalTok{()}
\end{Highlighting}
\end{Shaded}

\begin{figure}[H]
  \centering
  \includegraphics[width=0.375\linewidth]{_figures/programming/geom-mean-1-1}%
  \includegraphics[width=0.375\linewidth]{_figures/programming/geom-mean-1-2}
\end{figure}

If the list contains any \texttt{NULL} elements, they're ignored. This
makes it easy to conditionally add components:

\begin{Shaded}
\begin{Highlighting}[]
\NormalTok{geom_mean <-}\StringTok{ }\ControlFlowTok{function}\NormalTok{(}\DataTypeTok{se =} \OtherTok{TRUE}\NormalTok{) \{}
  \KeywordTok{list}\NormalTok{(}
    \KeywordTok{stat_summary}\NormalTok{(}\DataTypeTok{fun.y =} \StringTok{"mean"}\NormalTok{, }\DataTypeTok{geom =} \StringTok{"bar"}\NormalTok{, }\DataTypeTok{fill =} \StringTok{"grey70"}\NormalTok{),}
    \ControlFlowTok{if}\NormalTok{ (se) }
      \KeywordTok{stat_summary}\NormalTok{(}\DataTypeTok{fun.data =} \StringTok{"mean_cl_normal"}\NormalTok{, }\DataTypeTok{geom =} \StringTok{"errorbar"}\NormalTok{, }\DataTypeTok{width =} \FloatTok{0.4}\NormalTok{)}
\NormalTok{  )}
\NormalTok{\}}
\KeywordTok{ggplot}\NormalTok{(mpg, }\KeywordTok{aes}\NormalTok{(drv, cty)) }\OperatorTok{+}\StringTok{ }\KeywordTok{geom_mean}\NormalTok{()}
\KeywordTok{ggplot}\NormalTok{(mpg, }\KeywordTok{aes}\NormalTok{(drv, cty)) }\OperatorTok{+}\StringTok{ }\KeywordTok{geom_mean}\NormalTok{(}\DataTypeTok{se =} \OtherTok{FALSE}\NormalTok{)}
\end{Highlighting}
\end{Shaded}

\begin{figure}[H]
  \centering
  \includegraphics[width=0.375\linewidth]{_figures/programming/geom-mean-2-1}%
  \includegraphics[width=0.375\linewidth]{_figures/programming/geom-mean-2-2}
\end{figure}

\hypertarget{plot-components}{%
\subsection{Plot components}\label{plot-components}}

You're not just limited to adding layers in this way. You can also
include any of the following object types in the list:

\begin{itemize}
\item
  A data.frame, which will override the default dataset associated with
  the plot. (If you add a data frame by itself, you'll need to use
  \texttt{\%+\%}, but this is not necessary if the data frame is in a
  list.)
\item
  An \texttt{aes()} object, which will be combined with the existing
  default aesthetic mapping.
\item
  Scales, which override existing scales, with a warning if they've
  already been set by the user.
\item
  Coordinate systems and facetting specification, which override the
  existing settings.
\item
  Theme components, which override the specified components.
\end{itemize}

\hypertarget{annotation}{%
\subsection{Annotation}\label{annotation}}

It's often useful to add standard annotations to a plot. In this case,
your function will also set the data in the layer function, rather than
inheriting it from the plot. There are two other options that you should
set when you do this. These ensure that the layer is self-contained:
\index{Annotation!functions}

\begin{itemize}
\item
  \texttt{inherit.aes\ =\ FALSE} prevents the layer from inheriting
  aesthetics from the parent plot. This ensures your annotation works
  regardless of what else is on the plot. \indexc{inherit.aes}
\item
  \texttt{show.legend\ =\ FALSE} ensures that your annotation won't
  appear in the legend. \indexc{show.legend}
\end{itemize}

One example of this technique is the \texttt{borders()} function built
into ggplot2. It's designed to add map borders from one of the datasets
in the maps package: \indexf{borders}

\begin{Shaded}
\begin{Highlighting}[]
\NormalTok{borders <-}\StringTok{ }\ControlFlowTok{function}\NormalTok{(}\DataTypeTok{database =} \StringTok{"world"}\NormalTok{, }\DataTypeTok{regions =} \StringTok{"."}\NormalTok{, }\DataTypeTok{fill =} \OtherTok{NA}\NormalTok{, }
                    \DataTypeTok{colour =} \StringTok{"grey50"}\NormalTok{, ...) \{}
\NormalTok{  df <-}\StringTok{ }\KeywordTok{map_data}\NormalTok{(database, regions)}
  \KeywordTok{geom_polygon}\NormalTok{(}
    \KeywordTok{aes_}\NormalTok{(}\OperatorTok{~}\NormalTok{lat, }\OperatorTok{~}\NormalTok{long, }\DataTypeTok{group =} \OperatorTok{~}\NormalTok{group), }
    \DataTypeTok{data =}\NormalTok{ df, }\DataTypeTok{fill =}\NormalTok{ fill, }\DataTypeTok{colour =}\NormalTok{ colour, ..., }
    \DataTypeTok{inherit.aes =} \OtherTok{FALSE}\NormalTok{, }\DataTypeTok{show.legend =} \OtherTok{FALSE}
\NormalTok{  )}
\NormalTok{\}}
\end{Highlighting}
\end{Shaded}

\hypertarget{additional-arguments}{%
\subsection{Additional arguments}\label{additional-arguments}}

If you want to pass additional arguments to the components in your
function, \texttt{...} is no good: there's no way to direct different
arguments to different components. Instead, you'll need to think about
how you want your function to work, balancing the benefits of having one
function that does it all vs.~the cost of having a complex function
that's harder to understand. \indexc{...}

To get you started, here's one approach using \texttt{modifyList()} and
\texttt{do.call()}: \indexf{modifyList} \indexf{do.call}

\begin{Shaded}
\begin{Highlighting}[]
\NormalTok{geom_mean <-}\StringTok{ }\ControlFlowTok{function}\NormalTok{(..., }\DataTypeTok{bar.params =} \KeywordTok{list}\NormalTok{(), }\DataTypeTok{errorbar.params =} \KeywordTok{list}\NormalTok{()) \{}
\NormalTok{  params <-}\StringTok{ }\KeywordTok{list}\NormalTok{(...)}
\NormalTok{  bar.params <-}\StringTok{ }\KeywordTok{modifyList}\NormalTok{(params, bar.params)}
\NormalTok{  errorbar.params  <-}\StringTok{ }\KeywordTok{modifyList}\NormalTok{(params, errorbar.params)}
  
\NormalTok{  bar <-}\StringTok{ }\KeywordTok{do.call}\NormalTok{(}\StringTok{"stat_summary"}\NormalTok{, }\KeywordTok{modifyList}\NormalTok{(}
    \KeywordTok{list}\NormalTok{(}\DataTypeTok{fun.y =} \StringTok{"mean"}\NormalTok{, }\DataTypeTok{geom =} \StringTok{"bar"}\NormalTok{, }\DataTypeTok{fill =} \StringTok{"grey70"}\NormalTok{),}
\NormalTok{    bar.params)}
\NormalTok{  )}
\NormalTok{  errorbar <-}\StringTok{ }\KeywordTok{do.call}\NormalTok{(}\StringTok{"stat_summary"}\NormalTok{, }\KeywordTok{modifyList}\NormalTok{(}
    \KeywordTok{list}\NormalTok{(}\DataTypeTok{fun.data =} \StringTok{"mean_cl_normal"}\NormalTok{, }\DataTypeTok{geom =} \StringTok{"errorbar"}\NormalTok{, }\DataTypeTok{width =} \FloatTok{0.4}\NormalTok{),}
\NormalTok{    errorbar.params)}
\NormalTok{  )}

  \KeywordTok{list}\NormalTok{(bar, errorbar)}
\NormalTok{\}}

\KeywordTok{ggplot}\NormalTok{(mpg, }\KeywordTok{aes}\NormalTok{(class, cty)) }\OperatorTok{+}\StringTok{ }
\StringTok{  }\KeywordTok{geom_mean}\NormalTok{(}
    \DataTypeTok{colour =} \StringTok{"steelblue"}\NormalTok{,}
    \DataTypeTok{errorbar.params =} \KeywordTok{list}\NormalTok{(}\DataTypeTok{width =} \FloatTok{0.5}\NormalTok{, }\DataTypeTok{size =} \DecValTok{1}\NormalTok{)}
\NormalTok{  )}
\KeywordTok{ggplot}\NormalTok{(mpg, }\KeywordTok{aes}\NormalTok{(class, cty)) }\OperatorTok{+}\StringTok{ }
\StringTok{  }\KeywordTok{geom_mean}\NormalTok{(}
    \DataTypeTok{bar.params =} \KeywordTok{list}\NormalTok{(}\DataTypeTok{fill =} \StringTok{"steelblue"}\NormalTok{),}
    \DataTypeTok{errorbar.params =} \KeywordTok{list}\NormalTok{(}\DataTypeTok{colour =} \StringTok{"blue"}\NormalTok{)}
\NormalTok{  )}
\end{Highlighting}
\end{Shaded}

\begin{figure}[H]
  \centering
  \includegraphics[width=0.375\linewidth]{_figures/programming/unnamed-chunk-3-1}%
  \includegraphics[width=0.375\linewidth]{_figures/programming/unnamed-chunk-3-2}
\end{figure}

If you need more complex behaviour, it might be easier to create a
custom geom or stat. You can learn about that in the extending ggplot2
vignette included with the package. Read it by running
\texttt{vignette("extending-ggplot2")}.

\hypertarget{exercises-1}{%
\subsection{Exercises}\label{exercises-1}}

\begin{enumerate}
\def\labelenumi{\arabic{enumi}.}
\item
  To make the best use of space, many examples in this book hide the
  axes labels and legend. I've just copied-and-pasted the same code into
  multiple places, but it would make more sense to create a reusable
  function. What would that function look like?
\item
  Extend the \texttt{borders()} function to also add
  \texttt{coord\_quickmap()} to the plot.
\item
  Look through your own code. What combinations of geoms or scales do
  you use all the time? How could you extract the pattern into a
  reusable function?
\end{enumerate}

\hypertarget{sec:functions}{%
\section{Plot functions}\label{sec:functions}}

Creating small reusable components is most in line with the ggplot2
spirit: you can recombine them flexibly to create whatever plot you
want. But sometimes you're creating the same plot over and over again,
and you don't need that flexibility. Instead of creating components, you
might want to write a function that takes data and parameters and
returns a complete plot. \index{Plot functions}

For example, you could wrap up the complete code needed to make a
piechart:

\begin{Shaded}
\begin{Highlighting}[]
\NormalTok{piechart <-}\StringTok{ }\ControlFlowTok{function}\NormalTok{(data, mapping) \{}
  \KeywordTok{ggplot}\NormalTok{(data, mapping) }\OperatorTok{+}
\StringTok{    }\KeywordTok{geom_bar}\NormalTok{(}\DataTypeTok{width =} \DecValTok{1}\NormalTok{) }\OperatorTok{+}\StringTok{ }
\StringTok{    }\KeywordTok{coord_polar}\NormalTok{(}\DataTypeTok{theta =} \StringTok{"y"}\NormalTok{) }\OperatorTok{+}\StringTok{ }
\StringTok{    }\KeywordTok{xlab}\NormalTok{(}\OtherTok{NULL}\NormalTok{) }\OperatorTok{+}\StringTok{ }
\StringTok{    }\KeywordTok{ylab}\NormalTok{(}\OtherTok{NULL}\NormalTok{)}
\NormalTok{\}}
\KeywordTok{piechart}\NormalTok{(mpg, }\KeywordTok{aes}\NormalTok{(}\KeywordTok{factor}\NormalTok{(}\DecValTok{1}\NormalTok{), }\DataTypeTok{fill =}\NormalTok{ class))}
\end{Highlighting}
\end{Shaded}

\begin{figure}[H]
  \centering
  \includegraphics[width=0.5\linewidth]{_figures/programming/unnamed-chunk-4-1}
\end{figure}

This is much less flexible than the component based approach, but
equally, it's much more concise. Note that I was careful to return the
plot object, rather than printing it. That makes it possible add on
other ggplot2 components.

You can take a similar approach to drawing parallel coordinates plots
(PCPs). PCPs require a transformation of the data, so I recommend
writing two functions: one that does the transformation and one that
generates the plot. Keeping these two pieces separate makes life much
easier if you later want to reuse the same transformation for a
different visualisation. \index{Parallel coordinate plots}

\begin{Shaded}
\begin{Highlighting}[]
\NormalTok{pcp_data <-}\StringTok{ }\ControlFlowTok{function}\NormalTok{(df) \{}
\NormalTok{  is_numeric <-}\StringTok{ }\KeywordTok{vapply}\NormalTok{(df, is.numeric, }\KeywordTok{logical}\NormalTok{(}\DecValTok{1}\NormalTok{))}

  \CommentTok{# Rescale numeric columns}
\NormalTok{  rescale01 <-}\StringTok{ }\ControlFlowTok{function}\NormalTok{(x) \{}
\NormalTok{    rng <-}\StringTok{ }\KeywordTok{range}\NormalTok{(x, }\DataTypeTok{na.rm =} \OtherTok{TRUE}\NormalTok{)}
\NormalTok{    (x }\OperatorTok{-}\StringTok{ }\NormalTok{rng[}\DecValTok{1}\NormalTok{]) }\OperatorTok{/}\StringTok{ }\NormalTok{(rng[}\DecValTok{2}\NormalTok{] }\OperatorTok{-}\StringTok{ }\NormalTok{rng[}\DecValTok{1}\NormalTok{])}
\NormalTok{  \}}
\NormalTok{  df[is_numeric] <-}\StringTok{ }\KeywordTok{lapply}\NormalTok{(df[is_numeric], rescale01)}
  
  \CommentTok{# Add row identifier}
\NormalTok{  df}\OperatorTok{$}\NormalTok{.row <-}\StringTok{ }\KeywordTok{rownames}\NormalTok{(df)}
  
  \CommentTok{# Treat numerics as value (aka measure) variables}
  \CommentTok{# gather_ is the standard-evaluation version of gather, and}
  \CommentTok{# is usually easier to program with.}
\NormalTok{  tidyr}\OperatorTok{::}\KeywordTok{gather_}\NormalTok{(df, }\StringTok{"variable"}\NormalTok{, }\StringTok{"value"}\NormalTok{, }\KeywordTok{names}\NormalTok{(df)[is_numeric])}
\NormalTok{\}}
\NormalTok{pcp <-}\StringTok{ }\ControlFlowTok{function}\NormalTok{(df, ...) \{}
\NormalTok{  df <-}\StringTok{ }\KeywordTok{pcp_data}\NormalTok{(df)}
  \KeywordTok{ggplot}\NormalTok{(df, }\KeywordTok{aes}\NormalTok{(variable, value, }\DataTypeTok{group =}\NormalTok{ .row)) }\OperatorTok{+}\StringTok{ }\KeywordTok{geom_line}\NormalTok{(...)}
\NormalTok{\}}
\KeywordTok{pcp}\NormalTok{(mpg)}
\KeywordTok{pcp}\NormalTok{(mpg, }\KeywordTok{aes}\NormalTok{(}\DataTypeTok{colour =}\NormalTok{ drv))}
\end{Highlighting}
\end{Shaded}

\begin{figure}[H]
  \includegraphics[width=0.5\linewidth]{_figures/programming/pcp_data-1}%
  \includegraphics[width=0.5\linewidth]{_figures/programming/pcp_data-2}
\end{figure}

A complete exploration of this idea is \texttt{qplot()}, which provides
a fairly deep wrapper around the most common \texttt{ggplot()} options.
I recommend studying the source code if you want to see how far these
basic techniques can take you. \indexf{qplot}

\hypertarget{indirectly-referring-to-variables}{%
\subsection{Indirectly referring to
variables}\label{indirectly-referring-to-variables}}

The \texttt{piechart()} function above is a little unappealing because
it requires the user to know the exact \texttt{aes()} specification that
generates a pie chart. It would be more convenient if the user could
simply specify the name of the variable to plot. To do that you'll need
to learn a bit more about how \texttt{aes()} works.

\texttt{aes()} uses non-standard evaluation: rather than looking at the
values of its arguments, it looks at their expressions. This makes it
difficult to work with programmatically as there's no way to store the
name of a variable in an object and then refer to it later:

\begin{Shaded}
\begin{Highlighting}[]
\NormalTok{x_var <-}\StringTok{ "displ"}
\KeywordTok{aes}\NormalTok{(x_var)}
\CommentTok{#> * x -> x_var}
\end{Highlighting}
\end{Shaded}

Instead we need to use \texttt{aes\_()}, which uses regular evaluation.
There are two basic ways to create a mapping with \texttt{aes\_()}:
\indexf{aes\_}

\begin{itemize}
\item
  Using a \emph{quoted call}, created by \texttt{quote()},
  \texttt{substitute()}, \texttt{as.name()}, or \texttt{parse()}.
  \indexf{quote} \indexf{substitute} \indexf{parse} \indexf{as.name}

\begin{Shaded}
\begin{Highlighting}[]
\KeywordTok{aes_}\NormalTok{(}\KeywordTok{quote}\NormalTok{(displ))}
\CommentTok{#> * x -> displ}
\KeywordTok{aes_}\NormalTok{(}\KeywordTok{as.name}\NormalTok{(x_var))}
\CommentTok{#> * x -> displ}
\KeywordTok{aes_}\NormalTok{(}\KeywordTok{parse}\NormalTok{(}\DataTypeTok{text =}\NormalTok{ x_var)[[}\DecValTok{1}\NormalTok{]])}
\CommentTok{#> * x -> displ}

\NormalTok{f <-}\StringTok{ }\ControlFlowTok{function}\NormalTok{(x_var) \{}
  \KeywordTok{aes_}\NormalTok{(}\KeywordTok{substitute}\NormalTok{(x_var))}
\NormalTok{\}}
\KeywordTok{f}\NormalTok{(displ)}
\CommentTok{#> * x -> displ}
\end{Highlighting}
\end{Shaded}

  The difference between \texttt{as.name()} and \texttt{parse()} is
  subtle. If \texttt{x\_var} is ``a + b'', \texttt{as.name()} will turn
  it into a variable called
  \texttt{\textasciigrave{}a\ +\ b\textasciigrave{}}, \texttt{parse()}
  will turn it into the function call \texttt{a\ +\ b}. (If this is
  confusing, \url{http://adv-r.had.co.nz/Expressions.html} might help).
\item
  Using a formula, created with \texttt{\textasciitilde{}}.
  \indexc{\textasciitilde}

\begin{Shaded}
\begin{Highlighting}[]
\KeywordTok{aes_}\NormalTok{(}\OperatorTok{~}\NormalTok{displ)}
\CommentTok{#> * x -> displ}
\end{Highlighting}
\end{Shaded}
\end{itemize}

\texttt{aes\_()} gives us three options for how a user can supply
variables: as a string, as a formula, or as a bare expression. These
three options are illustrated below

\begin{Shaded}
\begin{Highlighting}[]
\NormalTok{piechart1 <-}\StringTok{ }\ControlFlowTok{function}\NormalTok{(data, var, ...) \{}
  \KeywordTok{piechart}\NormalTok{(data, }\KeywordTok{aes_}\NormalTok{(}\OperatorTok{~}\KeywordTok{factor}\NormalTok{(}\DecValTok{1}\NormalTok{), }\DataTypeTok{fill =} \KeywordTok{as.name}\NormalTok{(var)))}
\NormalTok{\}}
\KeywordTok{piechart1}\NormalTok{(mpg, }\StringTok{"class"}\NormalTok{) }\OperatorTok{+}\StringTok{ }\KeywordTok{theme}\NormalTok{(}\DataTypeTok{legend.position =} \StringTok{"none"}\NormalTok{)}

\NormalTok{piechart2 <-}\StringTok{ }\ControlFlowTok{function}\NormalTok{(data, var, ...) \{}
  \KeywordTok{piechart}\NormalTok{(data, }\KeywordTok{aes_}\NormalTok{(}\OperatorTok{~}\KeywordTok{factor}\NormalTok{(}\DecValTok{1}\NormalTok{), }\DataTypeTok{fill =}\NormalTok{ var))}
\NormalTok{\}}
\KeywordTok{piechart2}\NormalTok{(mpg, }\OperatorTok{~}\NormalTok{class) }\OperatorTok{+}\StringTok{ }\KeywordTok{theme}\NormalTok{(}\DataTypeTok{legend.position =} \StringTok{"none"}\NormalTok{)}

\NormalTok{piechart3 <-}\StringTok{ }\ControlFlowTok{function}\NormalTok{(data, var, ...) \{}
  \KeywordTok{piechart}\NormalTok{(data, }\KeywordTok{aes_}\NormalTok{(}\OperatorTok{~}\KeywordTok{factor}\NormalTok{(}\DecValTok{1}\NormalTok{), }\DataTypeTok{fill =} \KeywordTok{substitute}\NormalTok{(var)))}
\NormalTok{\}}
\KeywordTok{piechart3}\NormalTok{(mpg, class) }\OperatorTok{+}\StringTok{ }\KeywordTok{theme}\NormalTok{(}\DataTypeTok{legend.position =} \StringTok{"none"}\NormalTok{)}
\end{Highlighting}
\end{Shaded}

\begin{figure}[H]
  \includegraphics[width=0.333\linewidth]{_figures/programming/unnamed-chunk-8-1}%
  \includegraphics[width=0.333\linewidth]{_figures/programming/unnamed-chunk-8-2}%
  \includegraphics[width=0.333\linewidth]{_figures/programming/unnamed-chunk-8-3}
\end{figure}

There's another advantage to \texttt{aes\_()} over \texttt{aes()} if
you're writing ggplot2 plots inside a package: using
\texttt{aes\_(\textasciitilde{}x,\ \textasciitilde{}y)} instead of
\texttt{aes(x,\ y)} avoids the global variables NOTE in
\texttt{R\ CMD\ check}. \index{Global variables}

\hypertarget{the-plot-environment}{%
\subsection{The plot environment}\label{the-plot-environment}}

As you create more sophisticated plotting functions, you'll need to
understand a bit more about ggplot2's scoping rules. ggplot2 was written
well before I understood the full intricacies of non-standard
evaluation, so it has a rather simple scoping system. If a variable is
not found in the \texttt{data}, it is looked for in \emph{the} plot
environment. There is only one environment for a plot (not one for each
layer), and it is the environment in which \texttt{ggplot()} is called
from (i.e.~the \texttt{parent.frame()}). \index{Environments}
\indexf{parent.frame}

This means that the following function won't work because \texttt{n} is
not stored in an environment accessible when the expressions in
\texttt{aes()} are evaluated.

\begin{Shaded}
\begin{Highlighting}[]
\NormalTok{f <-}\StringTok{ }\ControlFlowTok{function}\NormalTok{() \{}
\NormalTok{  n <-}\StringTok{ }\DecValTok{10}
  \KeywordTok{geom_line}\NormalTok{(}\KeywordTok{aes}\NormalTok{(x }\OperatorTok{/}\StringTok{ }\NormalTok{n)) }
\NormalTok{\}}
\NormalTok{df <-}\StringTok{ }\KeywordTok{data.frame}\NormalTok{(}\DataTypeTok{x =} \DecValTok{1}\OperatorTok{:}\DecValTok{3}\NormalTok{, }\DataTypeTok{y =} \DecValTok{1}\OperatorTok{:}\DecValTok{3}\NormalTok{)}
\KeywordTok{ggplot}\NormalTok{(df, }\KeywordTok{aes}\NormalTok{(x, y)) }\OperatorTok{+}\StringTok{ }\KeywordTok{f}\NormalTok{()}
\CommentTok{#> Error in x/n: non-numeric argument to binary operator}
\end{Highlighting}
\end{Shaded}

Note that this is only a problem with the \texttt{mapping} argument. All
other arguments are evaluated immediately so their values (not a
reference to a name) are stored in the plot object. This means the
following function will work:

\begin{Shaded}
\begin{Highlighting}[]
\NormalTok{f <-}\StringTok{ }\ControlFlowTok{function}\NormalTok{() \{}
\NormalTok{  colour <-}\StringTok{ "blue"}
  \KeywordTok{geom_line}\NormalTok{(}\DataTypeTok{colour =}\NormalTok{ colour) }
\NormalTok{\}}
\KeywordTok{ggplot}\NormalTok{(df, }\KeywordTok{aes}\NormalTok{(x, y)) }\OperatorTok{+}\StringTok{ }\KeywordTok{f}\NormalTok{()}
\end{Highlighting}
\end{Shaded}

If you need to use a different environment for the plot, you can specify
it with the \texttt{environment} argument to \texttt{ggplot()}. You'll
need to do this if you're creating a plot function that takes user
provided data. See \texttt{qplot()} for an example.

\hypertarget{exercises-2}{%
\subsection{Exercises}\label{exercises-2}}

\begin{enumerate}
\def\labelenumi{\arabic{enumi}.}
\item
  Create a \texttt{distribution()} function specially designed for
  visualising continuous distributions. Allow the user to supply a
  dataset and the name of a variable to visualise. Let them choose
  between histograms, frequency polygons, and density plots. What other
  arguments might you want to include?
\item
  What additional arguments should \texttt{pcp()} take? What are the
  downsides of how \texttt{...} is used in the current code?
\item
  Advanced: why doesn't this code work? How can you fix it?

\begin{Shaded}
\begin{Highlighting}[]
\NormalTok{f <-}\StringTok{ }\ControlFlowTok{function}\NormalTok{() \{}
\NormalTok{  levs <-}\StringTok{ }\KeywordTok{c}\NormalTok{(}\StringTok{"2seater"}\NormalTok{, }\StringTok{"compact"}\NormalTok{, }\StringTok{"midsize"}\NormalTok{, }\StringTok{"minivan"}\NormalTok{, }\StringTok{"pickup"}\NormalTok{, }
    \StringTok{"subcompact"}\NormalTok{, }\StringTok{"suv"}\NormalTok{)}
  \KeywordTok{piechart3}\NormalTok{(mpg, }\KeywordTok{factor}\NormalTok{(class, }\DataTypeTok{levels =}\NormalTok{ levs))}
\NormalTok{\}}
\KeywordTok{f}\NormalTok{()}
\CommentTok{#> Error in factor(class, levels = levs): object 'levs' not found}
\end{Highlighting}
\end{Shaded}
\end{enumerate}

\hypertarget{functional-programming}{%
\section{Functional programming}\label{functional-programming}}

Since ggplot2 objects are just regular R objects, you can put them in a
list. This means you can apply all of R's great functional programming
tools. For example, if you wanted to add different geoms to the same
base plot, you could put them in a list and use \texttt{lapply()}.
\index{Functional programming} \indexf{lapply}

\begin{Shaded}
\begin{Highlighting}[]
\NormalTok{geoms <-}\StringTok{ }\KeywordTok{list}\NormalTok{(}
  \KeywordTok{geom_point}\NormalTok{(),}
  \KeywordTok{geom_boxplot}\NormalTok{(}\KeywordTok{aes}\NormalTok{(}\DataTypeTok{group =} \KeywordTok{cut_width}\NormalTok{(displ, }\DecValTok{1}\NormalTok{))),}
  \KeywordTok{list}\NormalTok{(}\KeywordTok{geom_point}\NormalTok{(), }\KeywordTok{geom_smooth}\NormalTok{())}
\NormalTok{)}

\NormalTok{p <-}\StringTok{ }\KeywordTok{ggplot}\NormalTok{(mpg, }\KeywordTok{aes}\NormalTok{(displ, hwy))}
\KeywordTok{lapply}\NormalTok{(geoms, }\ControlFlowTok{function}\NormalTok{(g) p }\OperatorTok{+}\StringTok{ }\NormalTok{g)}
\CommentTok{#> [[1]]}
\CommentTok{#> }
\CommentTok{#> [[2]]}
\CommentTok{#> }
\CommentTok{#> [[3]]}
\CommentTok{#> `geom_smooth()` using method = 'loess'}
\end{Highlighting}
\end{Shaded}

\begin{figure}[H]
  \includegraphics[width=0.333\linewidth]{_figures/programming/unnamed-chunk-12-1}%
  \includegraphics[width=0.333\linewidth]{_figures/programming/unnamed-chunk-12-2}%
  \includegraphics[width=0.333\linewidth]{_figures/programming/unnamed-chunk-12-3}
\end{figure}

If you're not familiar with functional programming, read through
\url{http://adv-r.had.co.nz/Functional-programming.html} and think about
how you might apply the techniques to your duplicated plotting code.

\hypertarget{exercises-3}{%
\subsection{Exercises}\label{exercises-3}}

\begin{enumerate}
\def\labelenumi{\arabic{enumi}.}
\item
  How could you add a \texttt{geom\_point()} layer to each element of
  the following list?

\begin{Shaded}
\begin{Highlighting}[]
\NormalTok{plots <-}\StringTok{ }\KeywordTok{list}\NormalTok{(}
  \KeywordTok{ggplot}\NormalTok{(mpg, }\KeywordTok{aes}\NormalTok{(displ, hwy)),}
  \KeywordTok{ggplot}\NormalTok{(diamonds, }\KeywordTok{aes}\NormalTok{(carat, price)),}
  \KeywordTok{ggplot}\NormalTok{(faithfuld, }\KeywordTok{aes}\NormalTok{(waiting, eruptions, }\DataTypeTok{size =}\NormalTok{ density))}
\NormalTok{)}
\end{Highlighting}
\end{Shaded}
\item
  What does the following function do? What's a better name for it?

\begin{Shaded}
\begin{Highlighting}[]
\NormalTok{mystery <-}\StringTok{ }\ControlFlowTok{function}\NormalTok{(...) \{}
  \KeywordTok{Reduce}\NormalTok{(}\StringTok{`}\DataTypeTok{+}\StringTok{`}\NormalTok{, }\KeywordTok{list}\NormalTok{(...), }\DataTypeTok{accumulate =} \OtherTok{TRUE}\NormalTok{)}
\NormalTok{\}}

\KeywordTok{mystery}\NormalTok{(}
  \KeywordTok{ggplot}\NormalTok{(mpg, }\KeywordTok{aes}\NormalTok{(displ, hwy)) }\OperatorTok{+}\StringTok{ }\KeywordTok{geom_point}\NormalTok{(), }
  \KeywordTok{geom_smooth}\NormalTok{(), }
  \KeywordTok{xlab}\NormalTok{(}\OtherTok{NULL}\NormalTok{), }
  \KeywordTok{ylab}\NormalTok{(}\OtherTok{NULL}\NormalTok{)}
\NormalTok{)}
\end{Highlighting}
\end{Shaded}
\end{enumerate}
